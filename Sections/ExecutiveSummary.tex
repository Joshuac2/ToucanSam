%\begin{singlespace}

The following design report represents the basis for a new aircraft developed by Team Toucan at the University of Illinois at Urbana-Champaign in response to AIAA's Request for Proposal \cite{RFP} (RFP) addressing the industry need for a high capacity, short to medium haul widebody capable of mitigating the increased traffic and congestion found at many modern airports.  While the design aircraft will be capable of carrying up to 400 passengers up to 3,500 NM, it has been engineered to bridge the gap between low capacity, short range regional jets and high capacity, long range twin jets currently on the market.  This report validates the possibility of addressing the prescribed requirements in a commercial aircraft.

 Prominent driving design points of the SAM Mark I include takeoff and landing distances under 9,000 ft, maximum range of 3,500 NM, capacity of 400 passengers and their respective luggage, 10 person crew (two pilots, eight flight attendants), adherence to FAA 14 CFR Part 25 regulations and fundamental desire to maximize customer value through minimizing fuel consumption as well as other operating costs.  The proposed aircraft will be service-ready by 2029 and has been designed as a twin engine, twin aisle widebody with a traditional tail.  Most notably, the entire wing including its sub-structure will be designed and manufactured utilizing modern composites to minimize structural weight while not sacrificing performance or longevity.  Other value-added design points include leading and trailing edge high lift devices, visual flight rule (VFR) and instrument flight rule (IFR) flight with autopilot, and de-icing capability for flight in less than ideal weather conditions.

\begin{table}[!h] 
    \centering
    \caption{At a Glance: Toucan Aviation's SAM Mark I}
    \begin{tabular}{ |c|c|c|c|c|c| }\toprule
    \textbf{Capacity} & \textbf{MTOW} & \textbf{Range} & \textbf{TOFL} & \textbf{LFL} & \textbf{Cruise Speed} \\\hline 
    400 Passengers & 457,390 [lb] & 3,500 [NM] & 9,000 [ft] & 9,000 [ft] & Mach 0.775  \\\hline \hline
    \textbf{SFC} & \textbf{TO Thrust} & \textbf{Wing Span} & \textbf{AR} & \textbf{MAC} & \textbf{Service Ceiling}  \\\hline 
    0.545 [lbm/lbf/hr] & 231,080 [lb] & 184 ft 5 in & 8.5 [$\sim$] & 25 ft 6 in & FL480  \\\hline \hline
    \textbf{Wing Loading} & \textbf{Thrust/Weight} & \textbf{Wing} & \textbf{Fuselage} & \textbf{Cost/Flight Hour} & \textbf{Flyaway Cost/1000}  \\\hline 
    114.35 [lb/ft$^2$] & 0.505 [$\sim$] & Composite & Metallic & \$82,379 [USD] & \$181,204,000 [USD] \\ \bottomrule

    \end{tabular}\label{tab:ataglance}
\end{table}

Important aircraft parameters are illustrated in Table \ref{tab:ataglance}.  These values have evolved from the original seed and fundamental based design after an in-depth analysis of their aerodynamic and performance implications.  Leading edge slats and trailing edge,  Fowler-type flaps are designed and integrated into the wing structure to efficiently generate lift under critical conditions such as take-off and landing.  Presentation of this final design follows a detailed finite element analysis (FEA) and computational fluid dynamics (CFD) analysis of the complete aircraft's aerodynamic surfaces and structure in order to ensure flight worthiness and is accompanied a thorough approximation of assembly cost, timeline, and life cycle maintenance costs.  A complete review of FAA 14 CFR Part 25 \cite{cfr} and the AIAA RFP \cite{RFP} has been completed to ensure delivery of a safe, value-driven aircraft for the AIAA 2020 undergraduate design competition.

%Considerable effort will continue to be put into meeting or exceeding applicable FAA 14 CFR Part 25 requirements for commercial aircraft, and final design will be completed by April 26, 2020 leaving adequate time for review, system integration check, FAR compliance check and submission of a safe, value-driven aircraft design by May 14th, 2020.

% \begin{table}[h!] 
%     \centering
%     \caption{Design Timeline $\&$ Team Waypoints}
%     \begin{tabular}{ |c||c| }\toprule
%     \textbf{Date} & \textbf{Task(s)} \\\hline\hline
%     3/30/2020 & Revision per FDR Comments Deadline \\\hline
%     4/6/2020 & Absolute Geometry Deadline \\ \hline
%     4/20/2020 & Computer-analysis/Derivation $\&$ Presentation Deadline \\ \hline
%     4/21/2020 & Final Design Presentation Deadline \\ \hline
%     4/26/2020 & Final Design Report Writing $\&$ Review Deadline \\ \hline
%     4/30/2020 & Final Design Submission \\ \hline
%     5/14/2020 & AIAA Submission \\\hline
%     \end{tabular}\label{deadlines}
% \end{table}

%\end{singlespace}

% \textcolor{red}{
% \begin{itemize}
%     \item Briefly discuss motivation for the aircraft, key requirements, and your design drivers. (JC \checkmark)
%     \item Briefly  summarize the proposed aircraft, including key  characteristics (e.g. general configuration, \textbf{TOGW, primary dimensions), key performance metrics, key differentiators, and a cost summary.} (JC \checkmark)
%     \item Include any requirements not met, struggling with, or have chosen to exceed. \textbf{(none?)}
%     \item Discuss the near term milestones and provide a project timeline for the semester. 
%     \item Maximum length: one page, single-spaced if necessary. (JC \checkmark)
% \end{itemize}}

%FROM DRR:
%Additionally, a component wise weight build up of corrected values for hybrid construction will supplement an in-depth stability and control analysis.  
