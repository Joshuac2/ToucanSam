The SAM Mk I has been developed to solve one of aviation's biggest challenges: congested airports in urban areas prone to significant delays.  As airlines continue to embrace the hub and spoke model, it becomes ever apparent modern aviation has ignored one of the biggest necessities in the market - a high capacity plane optimized to handle shorter routes while still retaining the capability of long-haul nonstop flight between the world's major destinations.  As a twin engine wide body, the SAM Mk I will be capable of comfortably carrying 400 passengers in a 50/350 two-class configuration with adequate range to fly transatlantic from the Eastern seaboard to popular European destinations, as well as many other routes within 3,500 NM.  Unlike many competitors, the SAM Mk I takes full advantage of contemporary composites and high-lift devices in order to feature a fully composite wing inclusive of the wing structure adjacent to traditional metallic fuselage, resulting in a significant increase in performance and added value to the end user.  While the aircraft is not a demonstration of the newest (and most expensive) technologies on the market, the design more importantly strikes the ever essential balance between cost and performance, the latter of which comes with significant expenditure of the former.  Additionally, this aircraft's value will only continue to increase as populous economies across the world mature, resulting in an impetus in demand on commercial aviation that not only fits within but also fully utilizes the existing infrastructure. 
\clearpage
% \textcolor{red}{
% \begin{itemize}
%     \item Discuss motivation, design objectives (e.g design drivers, key requirements), and a summary of key design characteristics and capabilities. \checkmark (\textit{JC})
%     \item This section should identify your niche (i.e. who your target customers are), convince the reader to read the rest of the report, and provide context for the remaining discussion.\checkmark (\textit{JC}) 
%     \item Discuss any unique attributes to your design philosophy. \checkmark (\textit{JC})
%     \item Start pagination as: 1, 2, 3,... \checkmark
%The importance of this design will only grow as the world's economy continues to mature, with developing economies representing a massive impetus to new aviation markets.  
%What else to mention 
% \end{itemize}}