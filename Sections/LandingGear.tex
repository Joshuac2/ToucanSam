\subsection{Configuration}
The aircraft will feature three sets of retractable landing gear customary of that found on similar aircraft within the aircraft featured in the trade study found in Table \ref{tab:trade_params}. Tentatively, the nose gear will be composed of two wheels, and the main gear will consist of two symmetric sets of six wheels each, three wheels in a line on each side of each set.  All of the gear will be hydraulically mounted to dampen the impact seen upon touchdown, as well as hydraulically actuated. The "tricycle" arrangement is traditionally seen due to both the stability of three on uneven surfaces as well as the weight distribution of the aircraft.  The front gear will also feature the main taxi light. Landing gear doors will most likely be utilized in order to increase performance characteristics by reducing drag. 

\subsection{Future Work -- Delete when discussed}
Future work will consist of an analysis of the load and stress placed on the landing gear during taxi, takeoff, and, most importantly, landing.  Additional consideration will be taken to ensure the landing gear and its related hydraulic systems stow within the contour of the fuselage as developed by aerodynamics. The necessary kinematics of the gear will be studied. Furthermore, brake sizing and fitment inside the wheel will be verified. The number, size, and type of wheels and number of struts needed must be finalized. Lastly, the height of the landing gear will be determined by the takeoff and flare angle, the placement of the gear, and length of the fuselage.

\subsection{Tire Selection (JJ)}
With the SAM Mark I falling within the weight and size boundaries of established commercial aircraft, designing to utilize tires already available from known manufacturers saves cost and time, and also mitigates the threat of setbacks due to developmental overruns from the manufacturer. Bridgestone was selected as a launch supplier for both the nose (two) and main (twelve) tires using existing tires from their APR line first introduced for the Boeing 777-300ER/200LR in 2004 \cite{bridgestonetire}.  Their listed specifications, as published by Bridgestone \cite{bridgestonetire}, are below in Table \ref{tab:tires}.  Per FAA FAR Part §25.733 \cite{cfr}, the tires will be filled with dry nitrogen to mitigate issues caused by dioxygen, a powerful oxidizer, at high heats as well as moisture, both found in ambient air. 

\begin{table}[!h]
    \centering
        \caption{Tire Specifications}
    \begin{tabular}{|c||c|c|c|c|c|c|}\toprule
         & \textbf{Size} & \textbf{Ply Rating} & \textbf{Speed Rating [MPH]} & \textbf{Rated Load [lb]} & \textbf{Average Weigh [lb]} & \textbf{Model} \\\hline \hline
         \textbf{Main} & 52x21.0R22 & 36 & 235 & 66,500 & 266 & APR07700 \\ \hline
         \textbf{Nose} & 43x17.5R17 & 32 & 235 & 44,500 & 156 & APR06600 \\ \hline
    \end{tabular}
    \label{tab:tires}
\end{table}


% \textcolor{red}{
% \begin{itemize}
%     \item Discuss landing gear sizing, tire sizing, loads, and retraction system.
%     \item Include CAD drawings with landing gear extended and stowed.
%     \item Discuss pressurization (if used).
%     \item Consider merging into Structures as a sub section
% \end{itemize}}